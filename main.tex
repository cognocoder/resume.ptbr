
\documentclass[letterpaper,11pt]{article}

\usepackage{latexsym}
\usepackage[empty]{fullpage}
\usepackage{titlesec}
\usepackage{marvosym}
\usepackage[usenames,dvipsnames]{color}
\usepackage{verbatim}
\usepackage{enumitem}
\usepackage[hidelinks]{hyperref}
\usepackage{fancyhdr}
\usepackage[english]{babel}
\usepackage{tabularx}


\pagestyle{fancy}
\fancyhf{} % clear all header and footer fields
\fancyfoot{}
\renewcommand{\headrulewidth}{0pt}
\renewcommand{\footrulewidth}{0pt}

% Adjust margins
\addtolength{\oddsidemargin}{-0.5in}
\addtolength{\evensidemargin}{-0.5in}
\addtolength{\textwidth}{1in}
\addtolength{\topmargin}{-.5in}
\addtolength{\textheight}{1.0in}

\urlstyle{same}

\raggedbottom
\raggedright
\setlength{\tabcolsep}{0in}

% Sections formatting
\titleformat{\section}{
  \vspace{-4pt}\scshape\raggedright\large
}{}{0em}{}[\color{black}\titlerule \vspace{-5pt}]

% Ensure that generate pdf is machine readable/ATS parsable
\pdfgentounicode=1

%-------------------------
% Custom commands
\newcommand{\resumeItem}[2]{
  \item\small{
    \textbf{#1}{: #2 \vspace{-2pt}}
  }
}

% Just in case someone needs a heading that does not need to be in a list
\newcommand{\resumeHeading}[4]{
    \begin{tabular*}{0.99\textwidth}[t]{l@{\extracolsep{\fill}}r}
      \textbf{#1} & #2 \\
      \textit{\small#3} & \textit{\small #4} \\
    \end{tabular*}\vspace{-5pt}
}

\newcommand{\resumeSubheading}[4]{
  \vspace{-1pt}\item
    \begin{tabular*}{0.97\textwidth}[t]{l@{\extracolsep{\fill}}r}
      \textbf{#1} & #2 \\
      \textit{\small#3} & \textit{\small #4} \\
    \end{tabular*}\vspace{-5pt}
}

\newcommand{\resumeSubSubheading}[2]{
    \begin{tabular*}{0.97\textwidth}{l@{\extracolsep{\fill}}r}
      \textit{\small#1} & \textit{\small #2} \\
    \end{tabular*}\vspace{-5pt}
}

\newcommand{\resumeSubItem}[2]{\resumeItem{#1}{#2}\vspace{-4pt}}

\renewcommand{\labelitemii}{$\circ$}

\newcommand{\resumeSubHeadingListStart}{\begin{itemize}[leftmargin=*]}
\newcommand{\resumeSubHeadingListEnd}{\end{itemize}}
\newcommand{\resumeItemListStart}{\begin{itemize}}
\newcommand{\resumeItemListEnd}{\end{itemize}\vspace{-5pt}}



\begin{document}

\begin{tabular*}{\textwidth}{l@{\extracolsep{\fill}}r}
  \textbf{{\Large Breno Martins da Costa Corrêa e Souza}} & 
    \href{mailto:breno.ec@gmail.com}{breno.ec@gmail.com} \\
  \href{https://www.linkedin.com/in/brenomccs}{LinkedIn \texttt{@brenomccs}} ·
  \href{https://github.com/cognocoder}{GitHub \texttt{@cognocoder}} &
    +55 31 99922 7282 \\
\end{tabular*}

\section{Educação}
  \resumeSubHeadingListStart
    \resumeSubheading
      {CEFET-MG -- Campus II}{Belo Horizonte, Minas Gerais}
      {Estudante de Mestrado em Modelagem Matemática e Computacional}{Fev. 2020 -- Presente}
      \resumeItemListStart
        \resumeItem{Desenvolvimento}{Ferramentas de simulação interativas e de
        computação em tempo real escritas em C++ e OpenGL, construídas por Make,
        CMake, testadas através do GoogleTest, documentadas por meio do Doxygen,
        versionadas com Git.}
        \resumeItem{Dissertação}{Texto tipografado em \LaTeX e diagramado
        através do Tikz, construído por meio do utilitário Make.}
      \resumeItemListEnd
    \resumeSubheading
      {CEFET-MG -- Campus II}{Belo Horizonte, Minas Gerais}
      {Bacharel em Engenharia de Computação}{Ago. 2009 -- Mar. 2015}
  \resumeSubHeadingListEnd

\section{Experiência}
  \resumeSubHeadingListStart
    \resumeSubheading
      {CEFET-MG -- Campus V}{Divinópolis, Minas Gerais}
      {Professor Substituto}{2018 -- 2019}
      \resumeItemListStart
        \resumeItem{Aulas}{Programação de Computadores em C/C++ e Java,
          Bancos de Dados Relacionais em SQL, dentre cursos técnicos e de
          graduação.}
        \resumeItem{Orientações}{Produção de trabalhos conclusivos e de jogo de
          cartas.}
        \resumeItem{Organização \& Arbitragem}{Estapas regionais e estaduais
          da Olimpíada Brasileira de Robótica.}
      \resumeItemListEnd

    \resumeSubheading
      {Mav Tecnologia}{Belo Horizonte, Minas Gerais}
      {Analista de Sistemas}{2014}
      \resumeItemListStart
        \resumeItem{Desenvolvimento Web}{Prototipação e planejamento para fusão
          de empresas com a Internet em Java, GWT, HTML, CSS e Python.}
      \resumeItemListEnd

    \resumeSubheading
      {Take Blip}{Belo Horizonte, Minas Gerais}
      {Estagiário em Pesquisa e Desenvolvimento}{2012 -- 2014}
      \resumeItemListStart
        \resumeItem{Desenvolvimento \textit{Full Stack}}{Serviços de Mensagens
          e de códigos promocionais em plataformas com interfaces SMS e Web em
          C\#, {.NET} Web Api, {.NET} MVC, Entity Framework, e SQL Server.
          Fui capaz de reduzir o número de registros em três ordens de magnitude
          em uma tabela de eventos através da exclusão de registros
          desnecessários e da simplificação de suas consultas de manutenção, que
          resultou em planos de execução de consultas e uso de armazenamento
          mais eficientes.}
      \resumeItemListEnd

    \resumeSubheading
      {DCSA/CEFET-MG -- Campus II}{Belo Horizonte, Minas Gerais}
      {Estagiário em Tecnologia da Informação}{2010 -- 2012}
      \resumeItemListStart
        \resumeItem{Desenvolvimento Web}{Manutenção do portal Web departamental
          e outros assuntos tecnologicos em Java Server Pages, JavaScript,
          HTML, CSS e OpenCMS.}
      \resumeItemListEnd
  \resumeSubHeadingListEnd

\section{Projetos}
  \resumeSubHeadingListStart
    \resumeSubItem{\href{https://cognocoder.github.io/coniventes/}{2020 $\cdot$ Coniventes}}
      {Projeto pessoal de mobilização com abordagem \textit{mobile first},
      hopedado no GitHub Pages, escrito em JavaScript, HTML e {CSS}.
      Conta com dados públicos consumidos em Node.js, mecanismo de pesquisa
      fuzzy e download de imagens sob demanda, para poupar dados móveis.
      Google Analytics regsitrou quase 2000 visitas únicas e próximo a 4000
      visualizações.}
  \resumeSubHeadingListEnd

\section{Publicações}
  \resumeSubHeadingListStart
    \resumeSubheading
      {CILAMCE 2017}{Florianópolis, Santa Catarina}
      {Speaker recognition with artificial neural networks}{2017}
  \resumeSubHeadingListEnd

\section{Programação \& Tecnologia}
  \resumeSubHeadingListStart
    \resumeSubItem{Intermediário}{%
      \textbf{HTML}, \textbf{CSS}, \textbf{JavaScript}, \textbf{C/C++},
      \textbf{CMake}, \textbf{Doxygen}, \textbf{Make}, \textbf{GoogleTest}, 
      \textbf{agile methodologies}, \textbf{Git}.}
    \resumeSubItem{Iniciante}{.NET, C\#, Java, MongoDB, Node.js, OpenGL, Python,
      PyTest, React, SQL.}
  \resumeSubHeadingListEnd

\end{document}
